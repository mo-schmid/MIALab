
%% bare_jrnl.tex
%% V1.4b
%% 2015/08/26
%% by Michael Shell
%% see http://www.michaelshell.org/
%% for current contact information.
%%
%% This is a skeleton file demonstrating the use of IEEEtran.cls
%% (requires IEEEtran.cls version 1.8b or later) with an IEEE
%% journal paper.
%%
%% Support sites:
%% http://www.michaelshell.org/tex/ieeetran/
%% http://www.ctan.org/pkg/ieeetran
%% and
%% http://www.ieee.org/

%%*************************************************************************
%% Legal Notice:
%% This code is offered as-is without any warranty either expressed or
%% implied; without even the implied warranty of MERCHANTABILITY or
%% FITNESS FOR A PARTICULAR PURPOSE! 
%% User assumes all risk.
%% In no event shall the IEEE or any contributor to this code be liable for
%% any damages or losses, including, but not limited to, incidental,
%% consequential, or any other damages, resulting from the use or misuse
%% of any information contained here.
%%
%% All comments are the opinions of their respective authors and are not
%% necessarily endorsed by the IEEE.
%%
%% This work is distributed under the LaTeX Project Public License (LPPL)
%% ( http://www.latex-project.org/ ) version 1.3, and may be freely used,
%% distributed and modified. A copy of the LPPL, version 1.3, is included
%% in the base LaTeX documentation of all distributions of LaTeX released
%% 2003/12/01 or later.
%% Retain all contribution notices and credits.
%% ** Modified files should be clearly indicated as such, including  **
%% ** renaming them and changing author support contact information. **
%%*************************************************************************


% *** Authors should verify (and, if needed, correct) their LaTeX system  ***
% *** with the testflow diagnostic prior to trusting their LaTeX platform ***
% *** with production work. The IEEE's font choices and paper sizes can   ***
% *** trigger bugs that do not appear when using other class files.       ***                          ***
% The testflow support page is at:
% http://www.michaelshell.org/tex/testflow/



\documentclass[journal]{IEEEtran}
%
% If IEEEtran.cls has not been installed into the LaTeX system files,
% manually specify the path to it like:
% \documentclass[journal]{../sty/IEEEtran}





% Some very useful LaTeX packages include:
% (uncomment the ones you want to load)


% *** MISC UTILITY PACKAGES ***
%
%\usepackage{ifpdf}
% Heiko Oberdiek's ifpdf.sty is very useful if you need conditional
% compilation based on whether the output is pdf or dvi.
% usage:
% \ifpdf
%   % pdf code
% \else
%   % dvi code
% \fi
% The latest version of ifpdf.sty can be obtained from:
% http://www.ctan.org/pkg/ifpdf
% Also, note that IEEEtran.cls V1.7 and later provides a builtin
% \ifCLASSINFOpdf conditional that works the same way.
% When switching from latex to pdflatex and vice-versa, the compiler may
% have to be run twice to clear warning/error messages.






% *** CITATION PACKAGES ***
%
\usepackage{cite}
% cite.sty was written by Donald Arseneau
% V1.6 and later of IEEEtran pre-defines the format of the cite.sty package
% \cite{} output to follow that of the IEEE. Loading the cite package will
% result in citation numbers being automatically sorted and properly
% "compressed/ranged". e.g., [1], [9], [2], [7], [5], [6] without using
% cite.sty will become [1], [2], [5]--[7], [9] using cite.sty. cite.sty's
% \cite will automatically add leading space, if needed. Use cite.sty's
% noadjust option (cite.sty V3.8 and later) if you want to turn this off
% such as if a citation ever needs to be enclosed in parenthesis.
% cite.sty is already installed on most LaTeX systems. Be sure and use
% version 5.0 (2009-03-20) and later if using hyperref.sty.
% The latest version can be obtained at:
% http://www.ctan.org/pkg/cite
% The documentation is contained in the cite.sty file itself.






% *** GRAPHICS RELATED PACKAGES ***
%
\ifCLASSINFOpdf
   \usepackage[pdftex]{graphicx}
  % declare the path(s) where your graphic files are
  \graphicspath{{../img/}}
  % and their extensions so you won't have to specify these with
  % every instance of \includegraphics
   \DeclareGraphicsExtensions{.pdf,.jpeg,.png}
\else
  % or other class option (dvipsone, dvipdf, if not using dvips). graphicx
  % will default to the driver specified in the system graphics.cfg if no
  % driver is specified.
  % \usepackage[dvips]{graphicx}
  % declare the path(s) where your graphic files are
  % \graphicspath{{../eps/}}
  % and their extensions so you won't have to specify these with
  % every instance of \includegraphics
  % \DeclareGraphicsExtensions{.eps}
\fi
% graphicx was written by David Carlisle and Sebastian Rahtz. It is
% required if you want graphics, photos, etc. graphicx.sty is already
% installed on most LaTeX systems. The latest version and documentation
% can be obtained at: 
% http://www.ctan.org/pkg/graphicx
% Another good source of documentation is "Using Imported Graphics in
% LaTeX2e" by Keith Reckdahl which can be found at:
% http://www.ctan.org/pkg/epslatex
%
% latex, and pdflatex in dvi mode, support graphics in encapsulated
% postscript (.eps) format. pdflatex in pdf mode supports graphics
% in .pdf, .jpeg, .png and .mps (metapost) formats. Users should ensure
% that all non-photo figures use a vector format (.eps, .pdf, .mps) and
% not a bitmapped formats (.jpeg, .png). The IEEE frowns on bitmapped formats
% which can result in "jaggedy"/blurry rendering of lines and letters as
% well as large increases in file sizes.
%
% You can find documentation about the pdfTeX application at:
% http://www.tug.org/applications/pdftex





% *** MATH PACKAGES ***
%
\usepackage{amsmath}
% A popular package from the American Mathematical Society that provides
% many useful and powerful commands for dealing with mathematics.
%
% Note that the amsmath package sets \interdisplaylinepenalty to 10000
% thus preventing page breaks from occurring within multiline equations. Use:
%\interdisplaylinepenalty=2500
% after loading amsmath to restore such page breaks as IEEEtran.cls normally
% does. amsmath.sty is already installed on most LaTeX systems. The latest
% version and documentation can be obtained at:
% http://www.ctan.org/pkg/amsmath





% *** SPECIALIZED LIST PACKAGES ***
%
%\usepackage{algorithmic}
% algorithmic.sty was written by Peter Williams and Rogerio Brito.
% This package provides an algorithmic environment fo describing algorithms.
% You can use the algorithmic environment in-text or within a figure
% environment to provide for a floating algorithm. Do NOT use the algorithm
% floating environment provided by algorithm.sty (by the same authors) or
% algorithm2e.sty (by Christophe Fiorio) as the IEEE does not use dedicated
% algorithm float types and packages that provide these will not provide
% correct IEEE style captions. The latest version and documentation of
% algorithmic.sty can be obtained at:
% http://www.ctan.org/pkg/algorithms
% Also of interest may be the (relatively newer and more customizable)
% algorithmicx.sty package by Szasz Janos:
% http://www.ctan.org/pkg/algorithmicx




% *** ALIGNMENT PACKAGES ***
%
%\usepackage{array}
\usepackage{changepage}
% Frank Mittelbach's and David Carlisle's array.sty patches and improves
% the standard LaTeX2e array and tabular environments to provide better
% appearance and additional user controls. As the default LaTeX2e table
% generation code is lacking to the point of almost being broken with
% respect to the quality of the end results, all users are strongly
% advised to use an enhanced (at the very least that provided by array.sty)
% set of table tools. array.sty is already installed on most systems. The
% latest version and documentation can be obtained at:
% http://www.ctan.org/pkg/array


% IEEEtran contains the IEEEeqnarray family of commands that can be used to
% generate multiline equations as well as matrices, tables, etc., of high
% quality.




% *** SUBFIGURE PACKAGES ***
\ifCLASSOPTIONcompsoc
\usepackage[caption=false,font=normalsize,labelfont=sf,textfont=sf]{subfig}
\else
\usepackage[caption=false,font=footnotesize]{subfig}
\fi
% subfig.sty, written by Steven Douglas Cochran, is the modern replacement
% for subfigure.sty, the latter of which is no longer maintained and is
% incompatible with some LaTeX packages including fixltx2e. However,
% subfig.sty requires and automatically loads Axel Sommerfeldt's caption.sty
% which will override IEEEtran.cls' handling of captions and this will result
% in non-IEEE style figure/table captions. To prevent this problem, be sure
% and invoke subfig.sty's "caption=false" package option (available since
% subfig.sty version 1.3, 2005/06/28) as this is will preserve IEEEtran.cls
% handling of captions.
% Note that the Computer Society format requires a larger sans serif font
% than the serif footnote size font used in traditional IEEE formatting
% and thus the need to invoke different subfig.sty package options depending
% on whether compsoc mode has been enabled.
%
% The latest version and documentation of subfig.sty can be obtained at:
% http://www.ctan.org/pkg/subfig




% *** FLOAT PACKAGES ***
%
%\usepackage{fixltx2e}
% fixltx2e, the successor to the earlier fix2col.sty, was written by
% Frank Mittelbach and David Carlisle. This package corrects a few problems
% in the LaTeX2e kernel, the most notable of which is that in current
% LaTeX2e releases, the ordering of single and double column floats is not
% guaranteed to be preserved. Thus, an unpatched LaTeX2e can allow a
% single column figure to be placed prior to an earlier double column
% figure.
% Be aware that LaTeX2e kernels dated 2015 and later have fixltx2e.sty's
% corrections already built into the system in which case a warning will
% be issued if an attempt is made to load fixltx2e.sty as it is no longer
% needed.
% The latest version and documentation can be found at:
% http://www.ctan.org/pkg/fixltx2e


%\usepackage{stfloats}
% stfloats.sty was written by Sigitas Tolusis. This package gives LaTeX2e
% the ability to do double column floats at the bottom of the page as well
% as the top. (e.g., "\begin{figure*}[!b]" is not normally possible in
% LaTeX2e). It also provides a command:
%\fnbelowfloat
% to enable the placement of footnotes below bottom floats (the standard
% LaTeX2e kernel puts them above bottom floats). This is an invasive package
% which rewrites many portions of the LaTeX2e float routines. It may not work
% with other packages that modify the LaTeX2e float routines. The latest
% version and documentation can be obtained at:
% http://www.ctan.org/pkg/stfloats
% Do not use the stfloats baselinefloat ability as the IEEE does not allow
% \baselineskip to stretch. Authors submitting work to the IEEE should note
% that the IEEE rarely uses double column equations and that authors should try
% to avoid such use. Do not be tempted to use the cuted.sty or midfloat.sty
% packages (also by Sigitas Tolusis) as the IEEE does not format its papers in
% such ways.
% Do not attempt to use stfloats with fixltx2e as they are incompatible.
% Instead, use Morten Hogholm'a dblfloatfix which combines the features
% of both fixltx2e and stfloats:
%
% \usepackage{dblfloatfix}
% The latest version can be found at:
% http://www.ctan.org/pkg/dblfloatfix

%\usepackage[ngerman]{babel}



%\ifCLASSOPTIONcaptionsoff
%  \usepackage[nomarkers]{endfloat}
% \let\MYoriglatexcaption\caption
% \renewcommand{\caption}[2][\relax]{\MYoriglatexcaption[#2]{#2}}
%\fi
% endfloat.sty was written by James Darrell McCauley, Jeff Goldberg and 
% Axel Sommerfeldt. This package may be useful when used in conjunction with 
% IEEEtran.cls'  captionsoff option. Some IEEE journals/societies require that
% submissions have lists of figures/tables at the end of the paper and that
% figures/tables without any captions are placed on a page by themselves at
% the end of the document. If needed, the draftcls IEEEtran class option or
% \CLASSINPUTbaselinestretch interface can be used to increase the line
% spacing as well. Be sure and use the nomarkers option of endfloat to
% prevent endfloat from "marking" where the figures would have been placed
% in the text. The two hack lines of code above are a slight modification of
% that suggested by in the endfloat docs (section 8.4.1) to ensure that
% the full captions always appear in the list of figures/tables - even if
% the user used the short optional argument of \caption[]{}.
% IEEE papers do not typically make use of \caption[]'s optional argument,
% so this should not be an issue. A similar trick can be used to disable
% captions of packages such as subfig.sty that lack options to turn off
% the subcaptions:
% For subfig.sty:
% \let\MYorigsubfloat\subfloat
% \renewcommand{\subfloat}[2][\relax]{\MYorigsubfloat[]{#2}}
% However, the above trick will not work if both optional arguments of
% the \subfloat command are used. Furthermore, there needs to be a
% description of each subfigure *somewhere* and endfloat does not add
% subfigure captions to its list of figures. Thus, the best approach is to
% avoid the use of subfigure captions (many IEEE journals avoid them anyway)
% and instead reference/explain all the subfigures within the main caption.
% The latest version of endfloat.sty and its documentation can obtained at:
% http://www.ctan.org/pkg/endfloat
%
% The IEEEtran \ifCLASSOPTIONcaptionsoff conditional can also be used
% later in the document, say, to conditionally put the References on a 
% page by themselves.




% *** PDF, URL AND HYPERLINK PACKAGES ***
%
%\usepackage{url}
% url.sty was written by Donald Arseneau. It provides better support for
% handling and breaking URLs. url.sty is already installed on most LaTeX
% systems. The latest version and documentation can be obtained at:
% http://www.ctan.org/pkg/url
% Basically, \url{my_url_here}.




% *** Do not adjust lengths that control margins, column widths, etc. ***
% *** Do not use packages that alter fonts (such as pslatex).         ***
% There should be no need to do such things with IEEEtran.cls V1.6 and later.
% (Unless specifically asked to do so by the journal or conference you plan
% to submit to, of course. )


% correct bad hyphenation here
\hyphenation{op-tical net-works semi-conduc-tor}


\begin{document}
%
% paper title
% Titles are generally capitalized except for words such as a, an, and, as,
% at, but, by, for, in, nor, of, on, or, the, to and up, which are usually
% not capitalized unless they are the first or last word of the title.
% Linebreaks \\ can be used within to get better formatting as desired.
% Do not put math or special symbols in the title.
\title{Automatic brain tissue segmentation in MR images using Random Forests and Conditional Random Fields}
%
%
% author names and IEEE memberships
% note positions of commas and nonbreaking spaces ( ~ ) LaTeX will not break
% a structure at a ~ so this keeps an author's name from being broken across
% two lines.
% use \thanks{} to gain access to the first footnote area
% a separate \thanks must be used for each paragraph as LaTeX2e's \thanks
% was not built to handle multiple paragraphs
%

\author{Marco~Lauria,~\IEEEmembership{Unibe,}
        Robert~Münger,~\IEEEmembership{Unibe,}
        and~Moritz~Schmid,~\IEEEmembership{Unibe}}% <-this % stops a space
%\thanks{M. Shell was with the Department
%of Electrical and Computer Engineering, Georgia Institute of Technology, Atlanta,
%GA, 30332 USA e-mail: (see http://www.michaelshell.org/contact.html).}% <-this % stops a space
%\thanks{J. Doe and J. Doe are with Anonymous University.}% <-this % stops a space
%\thanks{Manuscript received April 19, 2005; revised August 26, 2015.}}

% note the % following the last \IEEEmembership and also \thanks - 
% these prevent an unwanted space from occurring between the last author name
% and the end of the author line. i.e., if you had this:
% 
% \author{....lastname \thanks{...} \thanks{...} }
%                     ^------------^------------^----Do not want these spaces!
%
% a space would be appended to the last name and could cause every name on that
% line to be shifted left slightly. This is one of those "LaTeX things". For
% instance, "\textbf{A} \textbf{B}" will typeset as "A B" not "AB". To get
% "AB" then you have to do: "\textbf{A}\textbf{B}"
% \thanks is no different in this regard, so shield the last } of each \thanks
% that ends a line with a % and do not let a space in before the next \thanks.
% Spaces after \IEEEmembership other than the last one are OK (and needed) as
% you are supposed to have spaces between the names. For what it is worth,
% this is a minor point as most people would not even notice if the said evil
% space somehow managed to creep in.



% The paper headers
\markboth{Medical Image Analisys Lab, HS2020-2021}%
{Shell \MakeLowercase{\textit{et al.}}: Bare Demo of IEEEtran.cls for IEEE Journals}
% The only time the second header will appear is for the odd numbered pages
% after the title page when using the twoside option.
% 
% *** Note that you probably will NOT want to include the author's ***
% *** name in the headers of peer review papers.                   ***
% You can use \ifCLASSOPTIONpeerreview for conditional compilation here if
% you desire.




% If you want to put a publisher's ID mark on the page you can do it like
% this:
%\IEEEpubid{0000--0000/00\$00.00~\copyright~2015 IEEE}
% Remember, if you use this you must call \IEEEpubidadjcol in the second
% column for its text to clear the IEEEpubid mark.



% use for special paper notices
%\IEEEspecialpapernotice{(Invited Paper)}




% make the title area
\maketitle

% As a general rule, do not put math, special symbols or citations
% in the abstract or keywords.
%\begin{abstract}
%The abstract goes here.
%\end{abstract}


% Note that keywords are not normally used for peerreview papers.
%\begin{IEEEkeywords}
%Random Tree Search, Postprocessing, Probabilistic Keyhole Filling, Dense-CRF
%\end{IEEEkeywords}






% For peer review papers, you can put extra information on the cover
% page as needed:
% \ifCLASSOPTIONpeerreview
% \begin{center} \bfseries EDICS Category: 3-BBND \end{center}
% \fi
%
% For peerreview papers, this IEEEtran command inserts a page break and
% creates the second title. It will be ignored for other modes.
\IEEEpeerreviewmaketitle



\section{Introduction}
% The very first letter is a 2 line initial drop letter followed
% by the rest of the first word in caps.
% 
% form to use if the first word consists of a single letter:
% \IEEEPARstart{A}{demo} file is ....
% 
% form to use if you need the single drop letter followed by
% normal text (unknown if ever used by the IEEE):
% \IEEEPARstart{A}{}demo file is ....
% 
% Some journals put the first two words in caps:
% \IEEEPARstart{T}{his demo} file is ....
% 
% Here we have the typical use of a "T" for an initial drop letter
% and "HIS" in caps to complete the first word.

%\subsection{Subsection Heading Here}
%Subsection text here.
%
%% needed in second column of first page if using \IEEEpubid
%%\IEEEpubidadjcol
%
%\subsubsection{Subsubsection Heading Here}
%Subsubsection text here.


% An example of a floating figure using the graphicx package.
% Note that \label must occur AFTER (or within) \caption.
% For figures, \caption should occur after the \includegraphics.
% Note that IEEEtran v1.7 and later has special internal code that
% is designed to preserve the operation of \label within \caption
% even when the captionsoff option is in effect. However, because
% of issues like this, it may be the safest practice to put all your
% \label just after \caption rather than within \caption{}.
%
% Reminder: the "draftcls" or "draftclsnofoot", not "draft", class
% option should be used if it is desired that the figures are to be
% displayed while in draft mode.
%
%\begin{figure}[!t]
%\centering
%\includegraphics[width=2.5in]{myfigure}
% where an .eps filename suffix will be assumed under latex, 
% and a .pdf suffix will be assumed for pdflatex; or what has been declared
% via \DeclareGraphicsExtensions.
%\caption{Simulation results for the network.}
%\label{fig_sim}
%\end{figure}

% Note that the IEEE typically puts floats only at the top, even when this
% results in a large percentage of a column being occupied by floats.


% An example of a double column floating figure using two subfigures.
% (The subfig.sty package must be loaded for this to work.)
% The subfigure \label commands are set within each subfloat command,
% and the \label for the overall figure must come after \caption.
% \hfil is used as a separator to get equal spacing.
% Watch out that the combined width of all the subfigures on a 
% line do not exceed the text width or a line break will occur.
%
%\begin{figure*}[!t]
%\centering
%\subfloat[Case I]{\includegraphics[width=2.5in]{box}%
%\label{fig_first_case}}
%\hfil
%\subfloat[Case II]{\includegraphics[width=2.5in]{box}%
%\label{fig_second_case}}
%\caption{Simulation results for the network.}
%\label{fig_sim}
%\end{figure*}
%
% Note that often IEEE papers with subfigures do not employ subfigure
% captions (using the optional argument to \subfloat[]), but instead will
% reference/describe all of them (a), (b), etc., within the main caption.
% Be aware that for subfig.sty to generate the (a), (b), etc., subfigure
% labels, the optional argument to \subfloat must be present. If a
% subcaption is not desired, just leave its contents blank,
% e.g., \subfloat[].


% An example of a floating table. Note that, for IEEE style tables, the
% \caption command should come BEFORE the table and, given that table
% captions serve much like titles, are usually capitalized except for words
% such as a, an, and, as, at, but, by, for, in, nor, of, on, or, the, to
% and up, which are usually not capitalized unless they are the first or
% last word of the caption. Table text will default to \footnotesize as
% the IEEE normally uses this smaller font for tables.
% The \label must come after \caption as always.
%
%\begin{table}[!t]
%% increase table row spacing, adjust to taste
%\renewcommand{\arraystretch}{1.3}
% if using array.sty, it might be a good idea to tweak the value of
% \extrarowheight as needed to properly center the text within the cells
%\caption{An Example of a Table}
%\label{table_example}
%\centering
%% Some packages, such as MDW tools, offer better commands for making tables
%% than the plain LaTeX2e tabular which is used here.
%\begin{tabular}{|c||c|}
%\hline
%One & Two\\
%\hline
%Three & Four\\
%\hline
%\end{tabular}
%\end{table}


% Note that the IEEE does not put floats in the very first column
% - or typically anywhere on the first page for that matter. Also,
% in-text middle ("here") positioning is typically not used, but it
% is allowed and encouraged for Computer Society conferences (but
% not Computer Society journals). Most IEEE journals/conferences use
% top floats exclusively. 
% Note that, LaTeX2e, unlike IEEE journals/conferences, places
% footnotes above bottom floats. This can be corrected via the
% \fnbelowfloat command of the stfloats package.
\IEEEPARstart{P}{rogression} of neurodegenerative diseases can be tracked by the atrophy of brain tissues.
Manual segmentation and measurement is very time consuming and thus is not a viable option in clinical practice. 
The goal of the pipeline described in this paper is the segmentation of grey matter, white matter, hippocampus, amygdala and thalamus. 
Grey matter mostly consists of neuronal cell bodies, which are unmyelinated. White matter consists mostly of myelinated axons. These myelinated axons are connected to the grey matter areas. In the hippocampus occurs learning and memory. The amygdala is responsible for our emotions and aggression, and the thalamus is the relay centre for sensory information. To segment these parts correctly, i.e. know their position and volume accurately, is an important step during neurosurgical planning and simulation, which may lead to less complication and higher success rate during the surgery. 
In this paper we investigate the hypothesis whether spatial regularization improves the segmentation of the given segmentation pipeline. Therefore we have examined two different postprocessing approaches. Probabilistic keyhole filling to remove small isolated regions is a simple heuristic approach which is invented by us. In recent studies, postprocessing algorithm as conditional random field (CRF) achieved promising results. In a one of these studies fully connected CRF Model was used to postprocess segmented images of cities and landscape images \cite{Liu2018}. Other recent studies show the use of an efficient multi-scale 3D convolutional neural network to improve the segmentation \cite{Kamnitsas2017}. Hence, we propose improved segmentation of the parts described using a CRF algorithm as a second approach.
The chapter Material and Methods gives a quick overview of the whole pipeline. Our main focus is described in the sub-chapter postprocessing including its processes and procedures. The result section is divided by showing our accomplishments in each approach (PKF and CRF). These advantages and disadvantages, as well as our suggested procedure is described in the discussion/conclusion.

\section{Materials and Methods}
\subsection{Data}
The data used for the training of the model is from 30 unrelated healthy subjects from the Human Connectome Project data set. The data set contains images of a three Tesla MRI T1 and T2-weighted with segmentation as ground truth. The data set is defaced in order to make it  completely anonymous.
\subsection{Methods}
\begin{figure}[h!]
\centering
\includegraphics[width=2in]{img/pipeline}
\caption{Schematic representation of the image processing pipeline.}
\label{fig_pipeline}
\end{figure}


\subsubsection{Registration}

Alignment of the images to a common reference space named atlas.

\subsubsection{Preprocessing}

Improvement of the image quality for the subsequent classification.

\subsubsection{Feature Extraction}

Extraction of representing features for brain tissues.

\subsubsection{Classification}

Prediction of voxel label by a random forest classifier.

To obtain a baseline for the subsequent postprocessing, we perform a gridsearch for the random forest parameters $maximum\; tree\; depth$ and $number\; of\; estimators$. For the $maximum\; tree\; depth$ within the range 10 to 100 with step size 5 and for $number\; of\; estimators$ within the range 10 to 50 with step size 10. We achieved the following optimal parameters: $maximum\; tree\; depth = 85$, $number\; of\; estimators = 20$.
\subsubsection{Postprocessing}

The predicted masks with the segmentation have some obvious mistakes, e.g. keyholes, misclassified voxels and rough borders. Our approach to solving this issue is by using probabilistic keyhole filling and dense conditional random field algorithm. 
\newline
\textbf{Approach I: Probabilistic Keyhole Filling (PKF)}
\newline
In this section we explain the basic concept and the procedure of the PKF. The Goal of the PFK is to incorporate prior knowledge into the postprocessing. We know that we are segmenting connected structures within the brain, therefore the obtained labels also have to be connected. The structures to be segmented are also uniform in themselves.
This means that within the local neighbourhood (26-connected) of one voxel, all voxels should carry the same label (unless it is a boundary voxel).
To achieve this, in a first step, the probability images of the labels produced by the random forest algorithm are smoothed with a Gaussian filter. Subsequently, a new segmentation mask is created by assigning the label with the highest probability to a voxel. In the original segmentation mask generated by the pipeline, only those labels are assumed to be fixed which belong to the largest component of the respective label. All labels that are not fixed are then corrected using the newly generated segmentation mask. To better preserve areas that obviously belong to the background, there is the option to treat the background as an explicit label. The described algorithm thus has the variance of the Gaussian smoothing and the option "preserve background" as parameters.
\newline
\textbf{Approach II: Dense Conditional Random Field (DCRF)}
\newline
The DCRF model was first published by Philipp Krahenbühl and Vladlen Koltun in NIPS 2011 \cite{Krahenbuhl2012}. In Krähenbühl’s and Koltun’s paper, the exact derivation of the model can be seen. For this paper, the parameter appearance kernel and smoothness kernel were adjusted to achieve the best results . The appearance kernel takes nearby pixels with similar colours into account, which are most probable in the same segmentation label. The smoothness kernel removes small keyholes. The following equation \ref{eq:1} shows the influence of the adjustable parameter and weights of the two-kernel potentials:
\begin{multline}\label{eq:1}
    k\left(f_i,f_j\right)=\\
\omega^{\left(1\right)}  exp\left(-\frac{\left| p_i - p_j\right|^2}{2\theta^2_\alpha}-\frac{\left| I_i-I_j\right|^2}{2\theta^2_\beta}\right)\\
+\omega^{\left(2\right)}exp\left(-\frac{\left|p_i-p_j\right|^2}{2\theta^2_\gamma}\right)
\end{multline}

$\theta_\alpha$, $\theta_\beta$ for the appearance kernel and $\theta_\gamma$ for the smoothness kernel are the weights of the equation and were optimised using a grid search method. The higher these weights are, the less influence this kernel has in the DCRF model.
\subsubsection{Data Analysis}
The results are evaluated on whether the improvements are statistically significant (significance level $\alpha = 5\%$). Because the change of the metrics before and after the postprocessing is compared, the data is considered to be paired. Therefore, the tests are performed on the differences of the metrics before and after the postprocessing of a data set. Depending on the distribution of the obtained data a different statistical test is used. A paired, two-sided t-test is used if the distribution is approximately normal and a paired, two-sided sign test otherwise. The distribution is evaluated graphically using QQ-plots. The data is considered approximately normal distributed if all data points lie within the $95\%$ confidence bounds. The corresponding QQ-Plots can be found in Appendix \ref{sec_qq} QQ-Plot. For our data we used the paired two-sided t-test to evaluate the difference in the Dice coefficient and the sign test to evaluate the the difference in the Hausdorff distance. 
%\onecolumn
\section{Results}

\newcommand{\expnumber}[2]{{#1}\mathrm{e}{#2}}
\subsection{Probabilistic Keyhole Filling (PKF)}
The following three figures and the corresponding tables show the results of PKF with optimized parameter ($variance = 20$ and $preserve\; background = true$). In appendix \ref{sec_bb} the figures are shown again in full size.


\begin{figure}[ht]
\centering

%\includegraphics[width=3.4in]{img/boxplots/PKF-Dice.png}
\includegraphics[width=3.4in]{img/boxplots/PKF-Dice.png}


\caption{Comparison of the Dice coefficient before (no pp) and after postprocessing with PKF algorithm (with pp) for each label. The corresponding statistical data is shown in table \ref{tbl_pkf_dice}.
} 
\label{fig_pkf_dice}
\end{figure}

\begin{table}[ht]
%% increase table row spacing, adjust to taste
\renewcommand{\arraystretch}{1.4}
% if using array.sty, it might be a good idea to tweak the value of
%\extrarowheight %as needed to properly center the text within the cells
\caption{PKF: Statistical values of Dice coefficient}
\label{tbl_pkf_dice}
\centering
%% Some packages, such as MDW tools, offer better commands for making tables
%% than the plain LaTeX2e tabular which is used here.
\tabcolsep=0.11cm
\begin{tabular}{c|c|c|c|c|c}
\hline
 & WM & GM & H & A & T\\
\hline
$\mu$&$0.825$&$0.728$&$0.561$&$0.542$&$0.764$\\
%\hline
$\mu_{pkf}$&$0.824$&$0.728$&$0.570$&$0.547$&$0.773$\\
%\hline
$\Delta$&$\expnumber{-1.36}{-03}$&$\expnumber{-7.22}{-04}$&$\expnumber{8.75}{-03}$&$\expnumber{4.96}{-03}$&$9\expnumber{.16}{-03}$\\
%\hline
p&$	\expnumber{9.90}{-08}$&$\expnumber{1.96}{-04}$&$\expnumber{5.04}{-05}$&$	\expnumber{3.83}{-05}$&$\expnumber{1.64}{-05}$\\
%\hline
$CI_{low}$&$\expnumber{-1.56}{-03}$&$\expnumber{-9.93}{-04}$&$\expnumber{6.00}{-03}$&$	\expnumber{3.46}{-03}$&$\expnumber{6.67}{-03}$\\
%\hline
$CI_{high}$&$	\expnumber{-1.15}{-03}$&$\expnumber{-4.51}{-04}$&$\expnumber{1.15}{-02}$&$	\expnumber{6.47}{-03}$&$\expnumber{1.17}{-02}$\\
\hline
\multicolumn{6}{p{3.4in}}{Descriptive statistic and result of paired, two-sided t-test of Dice coefficient of PKF. $n=10$, significance level $\alpha = 5\%$. }\\
\multicolumn{6}{p{3.4in}}{$\mu=$ mean with no postprocessing; $\mu_{pkf}=$ mean with PKF postprocessing; $\Delta=\mu_{pkf}-\mu$; p $=$ p-value;  $CI_{low}= 95\%$ confidence interval lower bound; $CI_{high}= 95\%$ confidence interval upper bound.}\\
%\multicolumn{6}{p{135pt}}{$\Delta=\mu_{pkf}-\mu$; p $=$ p-value;  $CI_{low}= 95\%$ confidence interval lower bound; $CI_{high}= 95\%$ confidence interval upper bound.} \\

\end{tabular}
\end{table}


\begin{figure}[ht]
\centering

\includegraphics[width=3.4in]{img/boxplots/PKF-HD.png}
\caption{Comparison of the Hausdorff distance before (no pp) and after postprocessing with PKF algorithm (with pp) for each label. The corresponding statistical data is shown in table \ref{tbl_pkf_hd}.
}
\label{fig_pkf-hd}
\end{figure}

\begin{table}[ht]

%% increase table row spacing, adjust to taste
\renewcommand{\arraystretch}{1.3}
% if using array.sty, it might be a good idea to tweak the value of
%\extrarowheight %as needed to properly center the text within the cells
\caption{PKF: Statistical values of Hausdorff distance}
\label{tbl_pkf_hd}
\centering
%% Some packages, such as MDW tools, offer better commands for making tables
%% than the plain LaTeX2e tabular which is used here.
\tabcolsep=0.06cm
\begin{tabular}{c|c|c|c|c|c}
\hline
 & WM & GM & H & A & T\\
\hline
$m$ &2.639&2.000&12.470&12.980&14.877\\
$m_{pkf}$&2.236&2.000&11.467&12.490&5.193\\
$\Delta$&$\expnumber{-1.92}{-01}$&$\expnumber{0.00}{+00}$&$\expnumber{-1.00}{+00}$&$\expnumber{-6.47}{-01}$&$\expnumber{-9.14}{+00}$\\
p&$\expnumber{3.13}{-02}$&$\expnumber{5.00}{-01}$&$\expnumber{1.95}{-03}$&$\expnumber{1.95}{-03}$&$\expnumber{1.95}{-03}$\\
$CI_{low}$&$\expnumber{-5.23}{-01}$&$\expnumber{-1.16}{-01}$&$\expnumber{-1.42}{+00}$&$\expnumber{-9.32}{-01}$&$\expnumber{-1.09}{+01}$\\
$CI_{high}$&$\expnumber{0.00}{+00}$&$\expnumber{0.00}{+00}$&$\expnumber{-6.90}{-01}$&$\expnumber{-3.81}{-01}$&$\expnumber{-5.64}{+00}$\\
\hline
\multicolumn{6}{p{3.4in}}{Descriptive statistic and result of paired, two-sided sign test of Hausdorff distance of PKF. $n=10$, significance level $\alpha = 5\%$.}\\
\multicolumn{6}{p{3.4in}}{$m=$ median with no postprocessing; $m_{pkf}=$ median with PKF postprocessing; $\Delta=m_{pkf}-m$; p $=$ p-value;  $CI_{low}= 95\%$ confidence interval lower bound; $CI_{high}= 95\%$ confidence interval upper bound.}\\

\end{tabular}

\end{table}

\begin{figure}[h!]
\centering
%\includegraphics[width=3.4in]{img/boxplots/PKF.png}
\includegraphics[width=3.4in]{img/boxplots/PKF.png}
\caption{Graphical comparison of the axial ground truth (a) to axial segmentation with no postprocessing (b) and to the axial segmentation with PKF postprocessing method (c) }
\label{fig_pkf}
\end{figure}


In figure \ref{fig_pkf_dice} and the corresponding table \ref{tbl_pkf_dice} we can see that the PKF leads to changes of the mean of the Dice coefficient in the order of $10^-3$ or below. In the case of the labels White and Grey Matter it leads to a decrease and for the labels Hippocampus, Amygdala and Thalamus to an increase of the Dice coefficient. Although the changes are small, they are statistically significant, which can be seen in table \ref{tbl_pkf_dice}. In figure \ref{fig_pkf-hd} and the corresponding table \ref{tbl_pkf_hd} we can see that the PKF leads to an decrease of the Hausdorff distance for all labels except the label Grey Matter. Especially for the label Thalamus the Hausdorff distance is reduced by almost a factor of three. Except the label Grey Matter, the sign test on all labels indicates that all the changes are significant under a significance level $\alpha = 5\%$, as it is shown in table \ref{tbl_pkf_hd}. In figure \ref{fig_pkf} the visual improvement of the postprocessing can be seen. The red circle highlights the area where the miss labeled segmentation is corrected. Additionally the figure demonstrates that with the PKF method the overall uniformity of the segments is increased and less holes (Background) or miss labeled voxels can be found.

\subsection{Dense Conditional Random Field (DCRF)}
The following three figures and the corresponding tables show the results of DCRF with optimized parameter ($\theta_{\alpha} = 7$, $\theta_{\beta} = 3$ and $\theta_{\gamma} = 1$). In appendix \ref{sec_bb} the figures are shown again in full size.
\begin{figure}[ht]
\centering
%\includegraphics[width=3.4in]{img/boxplots/DCRF-DICE.png}
\includegraphics[width=3.4in]{img/boxplots/DCRF-DICE.png}
\caption{Comparison of the Dice coefficient before (no pp) and after postprocessing with DCRF algorithm (with pp) for each label. The corresponding statistical data is shown in table \ref{tbl_dcrf_dice}.}

\label{fig_dcrf_dice}
\end{figure}

\begin{table}[ht]
%% increase table row spacing, adjust to taste
\renewcommand{\arraystretch}{1.3}
% if using array.sty, it might be a good idea to tweak the value of
%\extrarowheight %as needed to properly center the text within the cells
\caption{DCRF: Statistical values of Dice coefficient}
\label{tbl_dcrf_dice}
\centering
%% Some packages, such as MDW tools, offer better commands for making tables
%% than the plain LaTeX2e tabular which is used here.
\tabcolsep=0.11cm
\begin{tabular}{c|c|c|c|c|c}
\hline
 & WM & GM & H & A & T\\
\hline
$\mu$&$0.824$&$0.726$&$0.564$&$0.560$&$0.753$\\
$\mu_{dcrf}$&$0.815$&$0.680$&$0.650$&$0.660$&$0.786$\\
$\Delta$&$\expnumber{-8.90}{-03}$&$\expnumber{-4.60}{-02}$&$\expnumber{8.58}{-02}$&$\expnumber{9.96}{-02}$&$\expnumber{3.33}{-02}$\\
p&$\expnumber{5.39}{-03}$&$\expnumber{4.57}{-12}$&$\expnumber{7.92}{-08}$&$\expnumber{1.16}{-07}$&$\expnumber{7.56}{-05}$\\
$CI_{low}$&$\expnumber{-1.44}{-02}$&$\expnumber{-4.82}{-02}$&$\expnumber{7.34}{-02}$&$\expnumber{8.45}{-02}$&$\expnumber{2.23}{-02}$\\
$CI_{high}$&$\expnumber{-3.37}{-03}$&$\expnumber{-4.38}{-02}$&$\expnumber{9.83}{-02}$&$\expnumber{1.15}{-01}$&$\expnumber{4.43}{-02}$\\
\hline
\multicolumn{6}{p{3.4in}}{Descriptive statistic and result of paired, two-sided t-test of Dice coefficient of DCRF. $n=10$, significance level $\alpha = 5\%$. }\\
\multicolumn{6}{p{3.4in}}{$\mu=$ mean with no postprocessing; $\mu_{dcrf}=$ mean with DCRF postprocessing; $\Delta=\mu_{dcrf}-\mu$; p $=$ p-value;  $CI_{low}= 95\%$ confidence interval lower bound; $CI_{high}= 95\%$ confidence interval upper bound.}\\

\end{tabular}
\end{table}



\begin{figure}[h!]
\centering
%\includegraphics[width=3.4in]{img/boxplots/DCRF-HD.png}
\includegraphics[width=3.4in]{img/boxplots/DCRF-HD.png}
\caption{Comparison of the Hausdorff distance before (no pp) and after postprocessing with DCRF algorithm (with pp) for each label. The corresponding statistical data is shown in table \ref{tbl_dcrf_hd}.}
\label{fig_dcrf-hd}
\end{figure}

\begin{table}[h!]

%% increase table row spacing, adjust to taste
\renewcommand{\arraystretch}{1.3}
% if using array.sty, it might be a good idea to tweak the value of
%\extrarowheight %as needed to properly center the text within the cells
\caption{DCRF: Statistical values of Hausdorff distance}
\label{tbl_dcrf_hd}
\centering
%% Some packages, such as MDW tools, offer better commands for making tables
%% than the plain LaTeX2e tabular which is used here.
\tabcolsep=0.08cm
\begin{tabular}{c|c|c|c|c|c}
\hline
 & WM & GM & H & A & T\\
\hline
$m$&$2.828$&$2.000$&$13.15$2&$12.44$8&$14.963$\\
$m_{dcrf}$&$2.236$&$3.000$&$6.632$&$10.93$0&$4.795$\\
$\Delta$&$\expnumber{-5.71}{-01}$&$\expnumber{8.28}{-01}$&$\expnumber{-6.16}{+00}$&$\expnumber{-1.14}{+00}$&$\expnumber{-1.05}{+01}$\\
p&$\expnumber{7.81}{-03}$&$\expnumber{1.95}{-03}$&$\expnumber{1.95}{-03}$&$\expnumber{1.95}{-03}$&$\expnumber{1.95}{-03}$\\
$CI_{low}$&$\expnumber{-7.08}{-01}$&$\expnumber{2.62}{-01}$&$\expnumber{-6.97}{+00}$&$\expnumber{-1.56}{+00}$&$\expnumber{-1.21}{+01}$\\
$CI_{high}$	&$\expnumber{6.92}{-02}$&$\expnumber{1.00}{+00}$&$\expnumber{-4.23}{+00}$&$\expnumber{-9.71}{-01}$&$\expnumber{-8.66}{+00}$\\
\hline
\multicolumn{6}{p{3.4in}}{Descriptive statistic and result of paired, two-sided sign test of Hausdorff distance of DCRF. $n=10$, significance level $\alpha = 5\%$.}\\
\multicolumn{6}{p{3.4in}}{$m=$ median with no postprocessing; $m_{dcrf}=$ median with PKF postprocessing; $\Delta=m_{dcrf}-m$; p $=$ p-value;  $CI_{low}= 95\%$ confidence interval lower bound; $CI_{high}= 95\%$ confidence interval upper bound.}\\

\end{tabular}

\end{table}

\begin{figure}[ht]
\centering

\includegraphics[width=3.4in]{img/boxplots/DCRF.png}
%\includegraphics[width=3.4in]{img/boxplots/DCRF.png}
\caption{Graphical comparison of the axial ground truth (a) to axial segmentation with no postprocessing (b) and to the axial segmentation with DCRF postprocessing method (c) }
\label{fig_dcrf}
\end{figure}

Figure \ref{fig_dcrf_dice} and the corresponding table \ref{tbl_dcrf_dice} show the that the DCRF algorithm results in a change of the Dice coefficient in the order of $10^-2$ or below. For the labels White and Grey Matter the postprocessing with DCRF leads to a decrease of the Dice coefficient. Whereas for the labels Hippocampus, Amygdala and Thalamus it leads to an increase of the Dice coefficient. The paired, two-sided t-test indicates that all of the changes are significant, as it is shown in table \ref{tbl_dcrf_dice}. In figure \ref{fig_dcrf-hd} and the corresponding table \ref{tbl_dcrf_hd} we can see, that the DCRF leads to a decrease of the Hausdorff distance for all labels except the label Grey Matter, where we can see an increase. Among all labels the largest decrease in the Hausdorff distance is obtained for the label Thalamus. The paired, two-sided sign test indicates that all the changes are significant under a significance level $\alpha = 5\%$, as we can see in table \ref{tbl_dcrf_hd}.

Figure \ref{fig_dcrf} shows the graphical difference of DCRF compared to the ground truth and no postprocessing. The improvement of the label Thalamus comes clearly forward, since again the red parts to the left and right of the Thalamus labeled regions are removed. These removed parts including some keyhole background regions are filled with the green label White Matter. 
The boundaries of the label Grey Matter are blown up, i.e. Grey Matter is less sparse in the postprocessing image using DCRF compared to the ground truth and the image without postprocessing.

%\newpage
%\twocolumn
\section{Discussion}
As shown in figure \ref{fig_pkf_dice}, the PKF algorithm has only a small influence in the range of $10^{-3}$ on the Dice coefficient. This could be due to the fact that only single voxels are relabeled by the algorithm, which does not strongly influence the Dice coefficient. In comparison, the DCRF algorithm produces a higher change in the Dice coefficient in the range of $10^{-1}$ at most. In contrast to the PKF, this algorithm optimizes entire segment ranges, which could explain the stronger change in the Dice coefficient.
As shown in the plots of the Hausdorff distance, the largest changes by the PKF as well as the DCRF are achieved in the labels Hippocampus, Amygdala and Thalamus. This is most likely due to the fact that before postprocessing all these labels contained incorrectly assigned voxels far away from the original structure. Because the PKF as well as the DCRF have the goal to correct incorrectly labeled voxels, the Hausdorff distance can be reduced and therefore improved by both approaches.
According to the statistical test, all differences are statistically significant except for the change in the Hausdorff distance by the PKF for the label White Matter. Despite this significance, the influence is not graphically evident in every case. Especially the changes of the Dice coefficients by the PKF seem to reflect a negligible influence. In contrast, the DCRF leads to much more uniform segments with smoother boundaries, which is most likely reflected in the larger changes in the Dice coefficient. The result of the reduced Hausdorff distance is clearly visible in both cases. It is reflected in the reduced number of incorrectly segmented regions compared between the segmentation without postprocessing and with postprocessing (e.g. figure \ref{fig_pkf} in the area outlined in red).


\section{Conclusion}
With respect to our hypothesis, we can conclude that spatial regularization can improve the segmentation of the given image processing pipeline. Both approaches are able to correct mislabelled voxels and thus lead to an improvement of the Hausdorff distance. In contrast to the PKF method, the DCRF is also able to influence the uniformity and the boundaries of the individual segments to a higher extent, which also leads to an improvement of the Dice coefficient.
In conclusion, we can say that for a rough improvement of mislabelled voxels a simple approach like PKF can be sufficient. If a finer and more sophisticated correction of the segments is required, a more complex method like DCRF should be used. With the disadvantage that the increased complexity of the enhancement requires a more elaborate fine-tuning of the parameters of DCRF.



% if have a single appendix:
%\appendix[Proof of the Zonklar Equations]
% or
%\appendix  % for no appendix heading
% do not use \section anymore after \appendix, only \section*
% is possibly needed

% use appendices with more than one appendix
% then use \section to start each appendix
% you must declare a \section before using any
% \subsection or using \label (\appendices by itself
% starts a section numbered zero.)
%

%\onecolumn
%\newpage
\appendices


% you can choose not to have a title for an appendix
% if you want by leaving the argument blank
%\section{}
%Appendix two text goes here.


% use section* for acknowledgment
%\twocolumn
\section*{Acknowledgment}

The work described in this paper was created in a group work during the lecture Medical Image Analysis Lab. The authors would like to thank Elias R\"ufenacht and You Suhang for the great support during the semester. The writing of this paper was created during the lecture Scientific Writing. The authors would like to thank Dr. Simon Milligan for the very useful workshops during the semester, which gave us plenty of recommendations regarding writing styles for scientific papers.
%\nocite{*}
\bibliographystyle{IEEEtran}
\bibliography{library}

\newpage
\onecolumn
\section{Boxplots}
\label{sec_bb}
In this section the boxplots are shown again but bigger. Our intention is to give the readers the possibility to read further statistical values more accurately. 
\subsection{PKF}
\label{sec_bb_pfk}
\begin{figure}[h!]
\centering
\includegraphics[width=4.9in]{img/boxplots/PKF-Dice.png}
\caption{PKF Dice coefficient compared to the Dice coefficient without postprocessing (no pp). The corresponding p-values are shown in the table \ref{tbl_pkf_dice}, next to the average Dice coefficients and the confidence intervals of each label.}
\label{fig_pkf_dice_appendix}
\end{figure}

\begin{figure}[h!]
\centering

\includegraphics[width=4.9in]{img/boxplots/PKF-HD.png}
\caption{PKF Hausdorff distance compared to the Hausdorff distance without postprocessing (no pp). The corresponding p-values are shown in the table \ref{tbl_pkf_hd}, next to the average Hausdorff distances and the confidence intervals of each label.}
\label{fig_pkf-hd_appendix}
\end{figure}

\begin{figure}[h!]
\centering

\includegraphics[width=4.9in]{img/boxplots/PKF.png}
\caption{Graphical comparison of the axial ground truth (a) to axial segmentation with no postprocessing (b) and to the axial segmentation with PKF postprocessing method (c) }
\label{fig_pkf_appendix}
\end{figure}
\newpage
\subsection{DCRF}
\label{sec_bb_dcrf}
\begin{figure}[h!]
\centering
\includegraphics[width=4.9in]{img/boxplots/DCRF-DICE.png}
\caption{DCRF Dice coefficient compared to the Dice coefficient without postprocessing (no pp)}
\label{fig_dcrf_dice_appendix}
\end{figure}

\begin{figure}[h!]
\centering
\includegraphics[width=4.9in]{img/boxplots/DCRF-HD.png}
\caption{DCRF Hausdorff distance compared to the Hausdorff distance without postprocessing (no pp)}
\label{fig_dcrf-hd_appendix}
\end{figure}
\begin{figure}[h!]
\centering

\includegraphics[width=4.9in]{img/boxplots/DCRF.png}
%\includegraphics[width=3.4in]{img/boxplots/DCRF.png}
\caption{Graphical comparison of the axial ground truth (a) to axial segmentation with no postprocessing (b) and to the axial segmentation with DCRF postprocessing method (c) }
\label{fig_dcrf_appendix}
\end{figure}

\clearpage
\newpage
\section{QQ-Plot}
\label{sec_qq}
These QQ-Plots helped us to chose the correct statistical test to evaluate the whether the the differences are statistical significant. 

\begin{figure}[hb]
\centering
\includegraphics[width=5.6in]{img/QQ_plots_PKF/GreyMatter.png}
\caption{QQ-Plot of the label Grey Matter}
\label{fig_qq_gm}
\end{figure}

\begin{figure}[ht]
\centering
\includegraphics[width=5.6in]{img/QQ_plots_PKF/WhiteMatter.png}
\caption{QQ-Plot of the label White Matter}
\label{fig_qq_wm}
\end{figure}

\begin{figure}[ht]
\centering
\includegraphics[width=5.6in]{img/QQ_plots_PKF/Hippocampus.png}
\caption{QQ-Plot of the label Hippocampus}
\label{fig_qq_h}
\end{figure}

\begin{figure}[ht]
\centering
\includegraphics[width=5.6in]{img/QQ_plots_PKF/Thalamus.png}
\caption{QQ-Plot of the label Thalamus}
\label{fig_qq_t}
\end{figure}

\begin{figure}[ht!]
\centering
\includegraphics[width=5.6in]{img/QQ_plots_PKF/Amygdala.png}
\caption{QQ-Plot of the label Amygdala}
\label{fig_qq_a}
\end{figure}


% Can use something like this to put references on a page
% by themselves when using endfloat and the captionsoff option.
\ifCLASSOPTIONcaptionsoff
  \newpage
\fi



% trigger a \newpage just before the given reference
% number - used to balance the columns on the last page
% adjust value as needed - may need to be readjusted if
% the document is modified later
%\IEEEtriggeratref{8}
% The "triggered" command can be changed if desired:
%\IEEEtriggercmd{\enlargethispage{-5in}}

% references section

% can use a bibliography generated by BibTeX as a .bbl file
% BibTeX documentation can be easily obtained at:
% http://mirror.ctan.org/biblio/bibtex/contrib/doc/
% The IEEEtran BibTeX style support page is at:
% http://www.michaelshell.org/tex/ieeetran/bibtex/

% argument is your BibTeX string definitions and bibliography database(s)

%
% <OR> manually copy in the resultant .bbl file
% set second argument of \begin to the number of references
% (used to reserve space for the reference number labels box)

%\nocite{*}
%\bibliographystyle{IEEEtran}
%\bibliography{library}
%\ibitem{IEEEhowto:kopka}


%H.~Kopka and P.~W. Daly, \emph{A Guide to \LaTeX}, 3rd~ed.\hskip 1em plus
%  0.5em minus 0.4em\relax Harlow, England: Addison-Wesley, 1999.
%\begin{thebibliography}
%\end{thebibliography}

% biography section
% 
% If you have an EPS/PDF photo (graphicx package needed) extra braces are
% needed around the contents of the optional argument to biography to prevent
% the LaTeX parser from getting confused when it sees the complicated
% \includegraphics command within an optional argument. (You could create
% your own custom macro containing the \includegraphics command to make things
% simpler here.)

% that's all folks
\end{document}


